\documentclass[professionalfont]{beamer}

\usepackage[T1]{fontenc}
\usepackage{amsmath}
\usepackage{graphicx}
\graphicspath{{figures/}}
\usepackage{booktabs}
\usepackage{amsmath,amssymb,mathtools}
\usepackage{multicol}
\usepackage{empheq}
\usepackage{tikz}
\usepackage[nodayofweek]{datetime} 
\usepackage[method=mhchem]{chemmacros}

\renewcommand{\dateseparator}{-}

\newcommand{\PTO}{\ce{PbTiO3}}

\newcommand{\PZT}{\ce{PbZr_{x}Ti_{1-x}O3}}

\newcommand{\Deg}{$^{\circ}$}
\newcommand{\degC}{$^{\circ}$C}
\newcommand{\Tc}{T$\mathrm{_{C}}$}

\newcommand{\TiIon}{\ce{Ti^{4+}}}
\newcommand{\OIon}{\ce{O^{2-}}}
\newcommand{\PbIon}{\ce{Pb^{2+}}}
\newcommand{\ZrIon}{\ce{Zr^{4+}}}
\newcommand{\TiOiPr}{Ti-o-\emph{i}-Pr}

\newcommand{\TMHD}{Pb(TMHD)$_{2}$}
\newcommand{\tmhd}{\TMHD}
\newcommand{\HFAc}{Pb(HFAc)$_{2}$}
\newcommand{\hfac}{\HFAc}
 
% usage: \tikzpic{<x percent>}{<y percent>}{<image size>}{<image name>}
\newcommand*{\tikzpic}[4]{%
\begin{tikzpicture}[remember picture,overlay]
\node at (current page.south west) [xshift=#1\paperwidth,yshift=#2\paperheight] {\includegraphics[width=#3\linewidth]{#4}};
\end{tikzpicture}%
}

\title[ALD of Complex Oxides]%
{\LARGE A Method for Atomic Layer Deposition\\
of Complex Oxide Thin Films}
\author[B. R. Beatty]{\underline{{\Large Brian R. Beatty}}\\[1em]%
				M.S. Thesis in Materials Science and Engineering}
\date[12/11/2012]{\small 11$^{\mathrm{th}}$ December, 2012}
\institute[Drexel University]{\large Drexel University, Philadelphia, PA}
\titlegraphic{\includegraphics[width=0.35\textwidth]{./Logos/mse_logo_large}}


\usetheme{POLIMI}
\setcounter{tocdepth}{1}

\usepackage{setspace}

\AtBeginSection[]
{
  \begin{frame}{Outline}
    \Large
    \tableofcontents[currentsection]
  \end{frame}
}

%%%%%%%%%%%%%%%%%%%%%%%%%%%%%%%%%
%%%%%%%%%%%%%%%%%%%%%%%%%%%%%%%%%

\begin{document}
\begin{frame}
\titlepage
\end{frame}

\begin{frame}
\frametitle{Outline}
%\begin{multicols}{2}
\Large
\tableofcontents
%\end{multicols}
\end{frame}

%%%%%%%%%%%%%%%%%%%%%%%

\section{Objectives}

\begin{frame}{Project Objectives}

\begin{itemize}
	\item Develop method for identifying best candidate precursors for depositing complex oxide films
	\vspace{1.5em}
	\item Determine optimal deposition parameters to obtain desired film stoichiometry
	\vspace{1.5em}
	\item Characterization of various film properties, for use in further optimizing subsequent depositions
	\vspace{1.5em}
	\item Successful deposition of desired material: \\Perovskite Lead Titanate (\ce{PbTiO3})
\end{itemize}

\end{frame}

%%%%%%%%%%%%%%%%%%%%%%%

\section{Atomic Layer Deposition}

%\begin{frame}{Atomic Layer Deposition}
%	\begin{itemize}
%		\item What is ALD?\vspace{1.5em}
%		\item What are its advantages and disadvantages?\vspace{1.5em}
%		\item Where is it used?
%	\end{itemize}
%\end{frame}

\begin{frame}{Atomic Layer Deposition}
	\begin{block}{What is ALD?}
		\begin{itemize}
			\item Chemical deposition method, similar to CVD
			\item Separation of deposition reaction into metal chemisorption and subsequent oxidation
			\item Restricts reactions to surface-vapor interactions, no vapor-vapor reactions possible
		\end{itemize}
	\end{block}
\end{frame}

\begin{frame}{Atomic Layer Deposition}
	\begin{overprint}
		\onslide<1>\centerline{\includegraphics[width=0.9\textwidth]{./Graphics/Synthesis/TMA1}}
		\onslide<2>\centerline{\includegraphics[width=0.9\textwidth]{./Graphics/Synthesis/TMA2}}
		\onslide<3>\centerline{\includegraphics[width=0.9\textwidth]{./Graphics/Synthesis/TMA3}}
		\onslide<4>\centerline{\includegraphics[width=0.9\textwidth]{./Graphics/Synthesis/TMA4}}
		\onslide<5>\centerline{\includegraphics[width=0.9\textwidth]{./Graphics/Synthesis/TMA5}}
		\onslide<6>\centerline{\includegraphics[width=0.9\textwidth]{./Graphics/Synthesis/TMA6}}
	\end{overprint}
\end{frame}

\begin{frame}{Atomic Layer Deposition}
	\vspace{-0.5cm}
	\begin{columns}[t]
		\column{0.5\textwidth}
			\begin{block}{Advantages}
				\begin{itemize}
					\item Ultra-high film thickness resolution (\AA-level)
					\item High film conformality\\(3D structure coating)
					\item Lower deposition temperatures
					\item Potentially lower environmental/economic impact
				\end{itemize}
			\end{block}
		\column{0.5\textwidth}
			\begin{block}{Disadvantages}
				\begin{itemize}
					\item Slow deposition rates
					\item Precursor chemistry is often difficult and complex (organometallic compounds)
					\item Many material systems lack developed ALD processes
				\end{itemize}
			\end{block}
	\end{columns}
\end{frame}

\begin{frame}{Atomic Layer Deposition}
	Where is ALD used?\vspace{1.5em}
	\begin{itemize}%[<+->]
		\item Integrated Circuits: Transistor Gate Oxides (high-k)\vspace{1.5em}
		\item Alternative Energy: Low tolerances for layer thickness, high film uniformity across surface
		\vspace{1.5em}
		\item Biomedical: Uniform coating of highly porous structures
	\end{itemize}
\end{frame}

%%%%%%%%%%%%%%%%%%%%%%%

\section{Thin Film Growth}

\subsection{Film Precursors}
\begin{frame}{Thin Film Growth: Film Precursors}
	\begin{columns}[c]
		\column{0.5\textwidth}
			\begin{block}{Titanium Precursor}
				Titanium(IV) tetraisopropoxide: {Ti-o-\emph{i} -Pr}
			\end{block}
			\vspace{0.5em}
			\begin{block}{Oxidizer}
				\begin{itemize}
					\item \ce{H2O} and \ce{O2}/\ce{O3} mixtures commonly used in literature
					\item \ce{O2}/\ce{O3} was chosen for higher compatibility with Pb precursors
				\end{itemize}
			\end{block}
		\column{0.5\textwidth}
			\begin{block}{Lead Precursors}
				\begin{enumerate}
					\item Bis(2,2,6,6-tetramethyl -3,5-heptanedionato) Lead(II): Pb(TMHD)$_{2}$
					\item Lead(II) hexafluoro- acetylacetonate: Pb(HFAc)$_{2}$
				\end{enumerate}
			\end{block}
	\end{columns}
\end{frame}

\subsection{Sample Substrates}
\begin{frame}{Thin Film Growth: Substrates}
\vfill
\begin{itemize}
	\large
	\item 200 nm \ce{SiO2}/Si(100)
	\begin{itemize}
		\item 200 nm of thermally grown silica on crystalline silica.
		\item Amorphous top layer
	\end{itemize}
	\vspace{1.5em}
	\item 15 nm Pt(111)/200 nm \ce{SiO2}/Si(100)
	\begin{itemize}
		\item 15 nm of ALD-grown platinum on the \ce{SiO2}/Si(100) substrate
		\item Metallic (crystalline) top layer
	\end{itemize}
	\vspace{1.5em}
	\item \ce{SrTiO3}(100)
	\begin{itemize}
		\item Single crystalline oriented strontium titanate wafer.
	\end{itemize}
	\vspace{1.5em}
\end{itemize}
\vfill
\end{frame}

\subsection{Deposition Parameters}
\begin{frame}{Thin Film Growth: Deposition Parameters}
\begin{itemize}
	\large
	\item Growth Temperature
	\vspace{1.5em}
	\item Precursor Dosage
	\vspace{1.5em}
	\item Purge Time
	\vspace{1.5em}
	\item Precursor Exposure
	\vspace{1.5em}
	\item Post-Deposition Annealing
	\end{itemize}
\end{frame}

%%%%%%%%%%%%%%%%%%%%%%%

\section{Characterization Methods}

\subsection{Thermal Analysis}
\begin{frame}
	\vspace{-0.5cm}
	\frametitle{Characterization Methods: Thermal Analysis}
	\begin{columns}[t]
		\column{0.5\textwidth}
			\begin{block}{Thermogravimetric Analysis}
				\begin{itemize}
					\item Method for analyzing mass loss rates as function of temperature
					\item Useful for determining optimal evaporation temperatures
					\item Can indicate multi-step evaporation/chemical conversion
				\end{itemize}
			\end{block}
		\column{0.5\textwidth}
			\begin{block}{Differential Calorimetry}
				\begin{itemize}
					\item Allows insight into energetic transformations as a function of temperature
					\item Indicates phase changes, evaporation energies, and structural changes
					\item Useful for analyzing the stability of precursors at desired temperatures
				\end{itemize}
			\end{block}
	\end{columns}
\end{frame}

\subsection{Composition Analysis}
\begin{frame}{Characterization Methods: Composition Analysis}

\begin{overprint}
	\onslide<1>%
		\vspace{0.5cm}
		\textbf{\large X-Ray Fluorescence Spectroscopy (XRF)}\vspace{1.5em}
		\begin{itemize}
			\item Similar to EDXS but uses X-rays in place of energetic electrons\vspace{1.5em}
			\item Much lower noise floor (no Bremsstrahlung radiation)\vspace{1.5em}
			\item Capable of quantitative compositional analysis of ultra-thin films
		\end{itemize}
	\onslide<2>\vspace{-0.5cm}\centerline{\includegraphics[width=0.9\textwidth]{./graphics/characterization/EDXS-scheme}}
\end{overprint}

\end{frame}

\subsection{Film Growth Rates}
\begin{frame}{Characterization Methods: Film Growth Rates}
\begin{overprint}
	\onslide<1>
		\vspace{0.5cm}
		\textbf{\large Ellipsometry}\vspace{1.5em}
		\begin{itemize}
			\item Non-destructive optical film analysis method\vspace{1.5em}
			\item Capable of determining numerous optical/electronic parameters of film\vspace{1.5em}
			\item Primarily used to determine post-deposition film thicknesses and thus growth rates
		\end{itemize}
	\onslide<2>
		\centerline{\includegraphics[width=\textwidth]{./graphics/characterization/ellipsometryDiagram_simple}}
\end{overprint}
\end{frame}

\subsection{Phase Identification}
\begin{frame}{Characterization Methods: Phase Identification}

\textbf{\large X-Ray Diffractometry (XRD)}
\vspace{1.5em}
\begin{itemize}
	\item Standard technique used to identify materials and phases/orientations
	\vspace{1.5em}
	\item Analysis produces information about presence of particular lattice spacings
	\vspace{1.5em}
	\item Identifying lattice spacings (via databases or previous studies in literature) can indicate presence and orientation of specific materials and phases 
\end{itemize}

\end{frame}

%%%%%%%%%%%%%%%%%%%%%%%

\section{Results}

\subsection{Thermal}
\begin{frame}{Results: Thermal Analysis}
\begin{overprint}
	\onslide<1>
		\begin{center}
%		\vspace{-0.5cm}
		DSC Cycles of \ce{Pb(HFAc)2} and \ce{Pb(TMHD)2}\\
		\vspace{0.5cm}
		\centerline{\includegraphics[width=0.45\textwidth]{./graphics/data/dsc/hfac}%
			\hspace{0.5cm}%
			\includegraphics[width=0.45\textwidth]{./graphics/data/dsc/tmhd}}
		\end{center}
	\onslide<2>
		\begin{center}
%		\vspace{-0.5cm}
		High-Temp. DSC Cycle of \ce{Pb(TMHD)2}\\
		\centerline{\includegraphics[width=0.6\textwidth]{./graphics/data/dsc/tmhd-300}}%
		\end{center}
	\onslide<3>
		\begin{center}
		TGA Traces for \ce{Pb(HFAc)2}\\
		\vspace{0.5cm}
		\centerline{\includegraphics[width=\textwidth]{./graphics/data/tga/hfac}}
		\end{center}
	\onslide<4>
		\begin{center}
		TGA Traces for \ce{Pb(TMHD)2}\\
		\vspace{0.5cm}
		\centerline{\includegraphics[width=\textwidth]{./graphics/data/tga/tmhd}}
		\end{center}
	\onslide<5>
		\begin{center}
		Constant Temperature Studies of \ce{Pb(HFAc)2} and \ce{Pb(TMHD)2}\\
		\vspace{0.5cm}
		\centerline{\includegraphics[width=\textwidth]{./graphics/data/tga/hold}}
		\end{center}		
\end{overprint}
\end{frame}

\subsection{Film Growth}

\begin{frame}{Results: Deposition and Processing Parameters}
	\vfill
	\begin{center}
	\begin{tabular}{ccccccc}
	\toprule
	&&&&&\multicolumn{2}{c}{Annealing}\\ \cmidrule{6-7}
	Temp.		&Run \#	&Pb:Ti	 &Cycles 	&Subs. 		&Temp. 		&Time \\ 
	(\degC{})		&		&Ratio	&		&Type		&(\degC{})	&(min) \\ \midrule%
	225			&20		&3:1		&200	&Si		 	&N/A			&N/A		\\
				&		&		&		&Pt-Si		&650		&90		\\
				&		&		&		&STO		&650		&90		\\
				&21		&3:1		&150	&Si		 	&N/A			&N/A		\\
				&		&		&		&Pt-Si		&650		&90		\\
				&		&		&		&STO		&650		&90		\\
				&22		&3:1		&150	&Si		 	&N/A			&N/A		\\
				&		&		&		&Pt-Si		&650		&90		\\
				&28		&3:1		&120	&STO		&650		&90		\\				
	\bottomrule
	\end{tabular}
	\end{center}
	\vfill
\end{frame}

\begin{frame}{Results: Film Growth Rates}
\begin{overprint}
	\onslide<1>
	\centerline{\includegraphics[width=0.7\textwidth]{./graphics/data/ellipsometry/Ellip-Rates}}
	\onslide<2>
	\vspace*{0.5cm}
	\begin{center}
	\begin{tabular}{cccc}
	\toprule
	Run \#	&Subs. 	&Thickness 	&Growth Rate 	\\ 
			&Type	&(nm)		&(\AA/cycle) 	\\ \midrule%
	20		&Si		&71.8		&3.59		\\
			&Pt-Si	&64.4		&3.22		\\
			&STO	&73.6		&3.68		\\
	21		&Si		&53.2		&3.54		\\
			&Pt-Si	&45.8		&3.05		\\
			&STO	&52.9		&3.53		\\
	22		&Si		&53.3		&3.55		\\
			&Pt-Si	&46.5		&3.10		\\
	28		&STO	&41.0		&3.42		\\				
	\bottomrule
	\end{tabular}
	\end{center}
	\vfill
\end{overprint}
\end{frame}

\subsection{Composition Analysis}
\begin{frame}{Results: Composition Analysis}
	\small
	\begin{center}
	\vspace{-0.5cm}
	Compositions of Selected Sample Films\\\vspace{0.5em}
	\begin{tabular}{l l r r r}
	\toprule
	&&\multicolumn{3}{c}{Composition (\%)}\\
	\cmidrule{3-5}
	Run \#&Substrate&Lead&Titanium&Ti:Pb Ratio\\
	\midrule
% 	Run	Sub-Type		Pb%		Ti%		Ti:Pb ratio
	20	&\ce{SiO2}	&56.6	&43.4	&0.769\\
		&Pt-Si		&51.5	&48.5	&0.944\\
	21	&\ce{SiO2}	&69.6	&30.4	&0.437\\
		&Pt-Si		&56.1	&43.9	&0.783\\
	22	&\ce{SiO2}	&67.7	&32.3	&0.478\\
		&Pt-Si		&56.1	&43.9	&0.784\\
	24	&\ce{SiO2}	&69.0	&31.0	&0.450\\
		&Pt-Si		&62.2	&37.8	&0.609\\
	\bottomrule
	\end{tabular}
	\end{center}
\end{frame}

\subsection{Phase Identification}
\begin{frame}{Results: Phase Identification}
\begin{overprint}
	\onslide<1>
		\begin{center}
		\textbf{\Large XRD of 20 on Pt-Si}\vspace{0.25cm}
		\centerline{\includegraphics[width=\textwidth]{./graphics/data/xrd/20Pt}}
		\end{center}
	\onslide<2>
		\begin{center}
		\textbf{\Large XRD of 23 on Pt-Si}
		\centerline{\includegraphics[width=0.95\textwidth]{./graphics/data/xrd/23Pt}}
		\end{center}
	\onslide<3>
		\begin{center}
		\textbf{\Large XRD of 28 on STO(100)}
		\centerline{\includegraphics[width=\textwidth]{./graphics/data/xrd/28STO}}
		\end{center}
\end{overprint}
\end{frame}

%%%%%%%%%%%%%%%%%%%%%%%

\section{Conclusions}

\begin{frame}{Conclusions}
\large
\begin{itemize}
\item A procedure for identifying best precursor compounds for ALD deposition of oxides was created
\vspace{1.5em}
\item A method for designing and implementing an ALD process for a novel material has been developed
\vspace{1.5em}
\item Successfully deposited thin films containing the target material: perovskite \ce{PbTiO3}
\vspace{1.5em}
\item Films contain significant amounts of impurity phases
\end{itemize}
\end{frame}

\subsection{Future Work}
\begin{frame}{Conclusions: Future Work}
\begin{itemize}
\item Refine process to maximize phase purity and film epitaxy
\vspace{2em}
\item Characterize ferroelectric character of crystallized films
\vspace{2em}
\item Investigate doping of thin films (e.g. \ce{PbZr_{x}Ti_{1-x}O3})
\vspace{2em}
\item Apply process to other oxide families (e.g. \ce{BaSrTiO3})
\end{itemize}
\end{frame}

%%%%%%%%%%%%%%%%%%%%%%%

\begin{frame}
	\vfill
	\begin{center}
		{\Huge Questions?}
	\end{center}
\end{frame}

\begin{frame}{Ellipsometry Model}
\begin{overprint}
\onslide<1>
\vfill
\begin{center}
{\bfseries\large Cauchy Model as used by WVASE32}\\
\begin{subequations}
\label{eq:cauchy}
\begin{align}
	n\left(\lambda\right) &= A_{n} + \frac{B_{n}}{\lambda^{2}}+\frac{C_{n}}{\lambda^{4}}+\cdots\\
        	\kappa\left(\lambda\right) &= A_{\kappa}e^{B_{\kappa}\left(\frac{hc}{\lambda}\right)-C_{\kappa}}
\end{align}
\end{subequations}
\end{center}
\vfill
\onslide<2>
\vfill
\begin{center}
{\bfseries\large Tauc-Lorentz Model as used by WVASE32}\\
\begin{subequations}
\label{eq:TaucLorentz}
\begin{align}
	\label{eq:TaucLorentz-1}
	\epsilon_{1}=\frac{2}{\pi}P\int^{\infty}_{E_{g}}\frac{\xi\epsilon_{2}\left(\xi\right)}{\xi^{2}-E^{2}}\,%
				\mathrm{d}\xi & \hspace{2.5cm}
\end{align}
\begin{empheq}[left=\empheqlbrace]{align}
	\epsilon_{2} (E) = \frac{AE_{0}C\left(E-E_{g}\right)^{2}}{\left(E^{2}-E^{2}_{0}\right)^{2}+C^{2}E^{2}}%
		\cdot \frac{1}{E} && E> E_{g}\\
        	\epsilon_{2} (E) = 0 && E\leq E_{g}
\end{empheq}
\end{subequations}
\end{center}
\vfill
\end{overprint}
\end{frame}

\begin{frame}{Ellipsometry Analysis Procedure}
\begin{overprint}
	\Large
	\onslide<1>
	\begin{enumerate}
	\item
	High-$\lambda$ Cauchy Model
	\item
	Direct Calculation of $n$ and $\kappa$
	\item
	Conversion to Oscillator Model
	\item
	Refinement of Oscillator Layer Parameters
	\end{enumerate}
	\onslide<2>
	\begin{itemize}
		\item High-$\lambda$ Cauchy Model
		\begin{itemize}
			\large
			\item Used to determine layer thickness
			\item Cauchy model applied to transparent region of film (>600nm)
		\end{itemize}
	\end{itemize}
	\onslide<3>
	\begin{itemize}
		\item Direct Calculation of $n$ and $\kappa$
		\begin{itemize}
			\large
			\item Numeric calculation of $n$ and $\kappa$ from Fresnel equations
			\item Used to provide a starting point to base Tauc-Lorentz oscillator model upon
			\item Calculated values are non-physical
		\end{itemize}
	\end{itemize}
	\onslide<4>
	\begin{itemize}
		\item Conversion to Oscillator Model
		\begin{itemize}
			\large
			\item Data set of $n$ and $\kappa$ are used to approximate initial guesses for %
			T-L oscillator parameters 
		\end{itemize}
	\end{itemize}
	\onslide<5>
	\begin{itemize}
		\item Refinement of Oscillator Layer Parameters
		\begin{itemize}
			\large
			\item Software optimization of model parameters
		\end{itemize}
	\end{itemize}
\end{overprint}
\end{frame}

\begin{frame}{Ellipsometry Model}
	\begin{center}
	\normalsize
	\vspace*{-1em}
	{\bfseries\large Model Parameters for 20-Pt/Si}\\[1em]
	\begin{tabular}{l l r r}
	\toprule
	Layer&Variable&Thickness (nm)&Value\\
	\midrule
	3. T-L Osc.&&75.6&\\
	&$\epsilon_{1}$ offset&&3.62\\
	&Amp&&36.54\\
	&E$_{\mathrm{n}}$&&4.51\\
	&C&&1.30\\
	&E$_{\mathrm{g}}$&&2.07\\
	2. \ce{Pt}&&15.1&\\
	1. \ce{SiO2}&& 1.1\\
	0. \ce{Si}&&Substrate&\\
	\bottomrule
	\end{tabular}
	\end{center}
\end{frame}

\begin{frame}{Raw Ellipsometry Data}
		\begin{center}
		\textbf{\large Ellipsometry Data from 20 - Pt/Si}\vspace{0.25cm}
		\centerline{\includegraphics[width=0.45\textwidth]{./graphics/data/ellipsometry/run-20-pt/psi}%
				 \hspace{1cm}%
				 \includegraphics[width=0.45\textwidth]{./graphics/data/ellipsometry/run-20-pt/delta}}
		\end{center}
\end{frame}

\begin{frame}{Ellipsometry Model}
	\begin{center}
	{\bfseries\large Model Parameters for 20-Pt/Si}\\[1em]
	\begin{tabular}{l l r r}
	\toprule
	Layer&Variable&Thickness (nm)&Value\\
	\midrule
	1. T-L Osc. (2)&&49.2&\\
	&$\epsilon_{1}$ offset&&1.42\\
	&Amp$_{1}$&&64.71\\
	&E$_{\mathrm{n 1}}$&&3.69\\
	&C$_{1}$&&4.44\\
	&E$_{\mathrm{g1}}$&&1.55\\
	&Amp$_{2}$&&1.55\\
	&E$_{\mathrm{n 2}}$&&2.12\\
	&C$_{2}$&&0.76\\
	&E$_{\mathrm{g2}}$&&0.001\\
	0. \ce{STO}&&Substrate&\\
	\bottomrule
	\end{tabular}
	\end{center}
\end{frame}

\begin{frame}{Raw Ellipsometry Data}
		\begin{center}
		\textbf{\large Ellipsometry Data from 28 - STO}\vspace{0.25cm}
		\centerline{\includegraphics[width=0.45\textwidth]{./graphics/data/ellipsometry/run-28-sto/psi}%
				 \hspace{1cm}%
				 \includegraphics[width=0.45\textwidth]{./graphics/data/ellipsometry/run-28-sto/delta}}
		\end{center}
\end{frame}

\begin{frame}{Ellipsometry Model}
	\begin{center}
	\vspace*{-1em}
	{\bfseries\large Model Parameters for 28-STO}\\[1em]
	\begin{tabular}{l l r r}
	\toprule
	Layer&Variable&Thickness (nm)&Value\\
	\midrule
	1. T-L Osc. (2)&&49.2&\\
	&$\epsilon_{1}$ offset&&1.42\\
	&Amp$_{1}$&&64.71\\
	&E$_{\mathrm{n 1}}$&&3.69\\
	&C$_{1}$&&4.44\\
	&E$_{\mathrm{g1}}$&&1.55\\
	&Amp$_{2}$&&1.55\\
	&E$_{\mathrm{n 2}}$&&2.12\\
	&C$_{2}$&&0.76\\
	&E$_{\mathrm{g2}}$&&0.001\\
	0. \ce{STO}&&Substrate&\\
	\bottomrule
	\end{tabular}
	\end{center}
\end{frame}

\begin{frame}
\end{frame}

\begin{frame}
\end{frame}



\end{document}












